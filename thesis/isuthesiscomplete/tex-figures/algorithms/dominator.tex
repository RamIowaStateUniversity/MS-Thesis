\begin{figure}[ht!]
% \centering
\begin{lstlisting}[numbers=left, tabsize=4, escapechar=@, caption={Dominator analysis: an example program analysis expressed using our
system.},label={lst:dominators}] 
allNodes : Set<int>;

initT := traversal (n : Node) { 
	add(allNodes, n.id);
}

domT := traversal (n : Node) : Set<int> { 
	Set<int> dom;
	if(output(n, domT) == null)
		dom = allNodes;
	else
		dom = output(n, domT);
	foreach(p : n.preds) 
		dom = intersection(dom, output(p, domT)) 
	add(dom, n.id); 
	return dom; 
} 

fp := fixp(Set<int> curr, Set<int> prev) : bool {
	if(equals(curr, prev))
		return true;
	return false;
}

traverse(g, initT, ITERATIVE); @\label{line:traverse}@
traverse(g, domT, FORWARD, fp); 				
\end{lstlisting}
% \caption{}
% % 
% % 
% % Dominator analysis performs traversal over the graph to determine the
% % dominators of each node. It has two traversals. \textit{init} traversal collects
% % all node ids. \textit{dom\_T} is called with a fixpoint and it runs till all the
% % nodes reach the fixpoint condition.}
% \label{fig:dominator}
\end{figure}


