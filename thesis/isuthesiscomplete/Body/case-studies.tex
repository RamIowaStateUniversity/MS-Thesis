\chapter{Case Studies}
\label{sec:case-studies}

\begin{figure}
%\begin{figure}[t]%
\centering
\scriptsize
\setlength{\tabcolsep}{4pt}
\begin{tabular}{lrrr}
\toprule
Case & Hybrid & WRPO & Reduce \\
\midrule
APM & 1527 & 1702 & 10\% \\
AUM & 883 & 963 & 8\% \\
SVT & 1417 & 1501 & 6\% \\
\bottomrule
\end{tabular}%
\caption{Running time (minutes) of the case studies on GitHub data.}
\label{fig:case-study-table}
%\end{figure}
\end{figure}

We implemented three case studies using our formalism for source 
code analysis and evaluated using hybrid and WRPO traversal. We are comparing 
against only WRPO since it is the next best performing traversal. 
%for all the three case studies.

\textbf{API Precondition Mining (APM).} This case study mines a large corpus 
of API usages to derive potential preconditions for API methods~\cite{nguyen2014mining}. 
The key idea of this work is that API preconditions would be checked frequently in a 
corpus with a large number of API usages, while project-specific 
conditions would be less frequent. 
%
This case study analysis mined the preconditions for all methods of 
\lstinline|java.lang.String|. 

\textbf{API Usage Mining (AUM).} This case study analyzes API usage code and 
mines API usage patterns~\cite{zhong2009mapo}. The mined patterns help 
developers understand and write API usages more effectively 
with less errors. Our analysis mined usage patterns for 
\lstinline|java.util| APIs. 

\textbf{Finding Security Vulnerabilities with Tainted Object 
Propagation (SVT).} This case study formulated a variety of widespread SQL 
injections, as tainted object propagation problems~\cite{livshits2005finding}. 
Our analysis looked for all SQL injection vulnerabilities 
matching the specifications in the statically analyzed code. 

\autoref{fig:case-study-table} shows that hybrid traversal helps reduce 
running times significantly by 80--175 minutes, which is from 6\%--10\% relatively. 

%\input{tex-figures/precondition-result}
%
%\autoref{fig:precondition-result} shows the first and second most occurring 
%preconditions for top-4 most frequently-used methods in \lstinline|java.lang.String|. 
%%
%All of them have only one argument. {\receiver} and {\argument} denote the 
%receiver and the argument, respectively. All the first conditions correctly 
%capture the constraint of non-nullability on the receivers of those instant 
%methods. 
