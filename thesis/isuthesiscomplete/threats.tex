\chapter{Threats to Validity}
\label{sec:threats}
% \todo{1. Second paragraph had data about branches along with loops - add that back please.\newline
% 2. check this sentence\newline
% "role in improving the overall performance of the analysis a large dataset of graphs."}\par
Our first threat to validity is our selection of source code analysis used in our
evaluation. While there exists no source for a standard set of analysis, we
relied mainly on text books and source code analysis tools to select analysis. We
have selected basic control and data-flow analyses, and analyses to find bugs or
code smells.
We made sure to include analysis that covers all the properties of interest. For
instance, our analysis set includes: both forward/backward analysis, data-flow sensitive and insensitive analysis, loop sensitive and insensitive
analysis. 
% As part of future work, we plan to extend the scope to
% inter-procedural analyses, which should enable more sophisticated choices like
% pointer and alias analysis.

Our next threat to validity is our selection of ultra-large-scale datasets that
provide graphs for running the analyses. The datasets do not contain a balanced
distribution of different graph \graphprop{} (sequential, branch and loop). Both
DaCapo and GitHub datasets contains majority of sequential graphs (65\% and
69\%, respectively) and only 10\% are graphs with loops. The impact of this
threat can be seen in our evaluation of the importance of paths and decisions in
our decision tree. Paths and decisions along sequential graphs are taken more
often. This threat is not easy to mitigate, as it is hard to find and difficult
to expect a real-world code dataset to contain a balanced distribution of graphs
of various types. Nonetheless, our evaluation shows that the selection and
optimization of the best traversal strategy for these 35\% of the graphs (graphs
with branches and loops) plays an important role in improving the overall
performance of the analysis over a large dataset of graphs.
% Possible threats:
% \begin{itemize}
% 	\item Analysis
% 	\begin{itemize}
% 		\item Selected analysis contains only 4 backward traversals when compared to 11 forward traversals. So the decisions and reductions will be slightly skewed towards traversal strategies favoring forward traversals.
% 		\item Only two analysis with loop insensitive property were included in the traversal which undermines the decisions favoring loop insensitive traversals.
% 	\end{itemize}
% 	\item Running time
% 	\begin{itemize}
% 	\item All analyses were run only once as running multiple times on such large dataset is not feasible.
% 	\end{itemize}
% 	\item Dataset
% 	\begin{itemize}
% 	\item Both the dataset contains only 10\% of cfg with loops. This once again undermines the decisions taken for cfgs with loops.
% 	\end{itemize}
% 	\item Traversal implementation (Not sure whether this is a threat)
% 	\begin{itemize}
% 	\item Traversals were implemented as optimized as possible. But there could be more optimized version out there which if used can alter the results.
% 	\end{itemize}
% \end{itemize}